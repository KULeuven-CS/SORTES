\documentclass[../main.tex]{subfiles} 

\begin{document}
\chapter{Implementing ASG}
In this chapter we look at how AS can be implemented on different architectures.

\section{Implementing ASG on a Naked Computer}
The proposed implementation is intended for rather simple systems running on a single processor. It can, of course be elaborated to suit more complex systems.
\subsection{Scheduling Parallel Components}
A naive way to schedule a set of tasks that may be appropriate in simple systems is the cooperative multitasking loop.
This is simply having a main loop run with the different tasks in it.
Depending on the relative frequency the task should run the task is called multiple times.
This kind of scheduling is non premptive meaning the OS will never interrupt a running task.
Because of this each task should be as short as possible.
The worse case respone time will be the sum of the maximum execution times of the functions.
A simple code example can be seen in \ref{ex:coop_multi}.

\begin{lstlisting}[style=cstyle, caption=Example of cooperative multitasking where task A is run twice as frequent as task B. ,label=ex:coop_multi]
	while(1){
		TaskB();
		TaskA();
		TaskC();
		TaskB();
		TaskD();
	}
\end{lstlisting}

\subsection{Implementing Barriers}
\subsection{Implementing Resource Management}
\subsection{Handling MINVT \& Timeouts}

\section{Implementating ASG on MicroC/OS-II}

\subsection{Scheduling Parallel Components}
\subsection{Implementing Barriers}
\subsection{Implementing Resource Management}
\subsection{Handling MINVT \& Timeouts}

\end{document}
